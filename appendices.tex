\chapter{DOCUMENTAÇÃO CASO DE USO}
\label{Apx:A}


\begin{longtable}{|l|l|}
\caption{Documentação Caso de Uso.} \label{tab:long} \\

\hline 
\multicolumn{1}{|c|}{\textbf{Nome do Caso de Uso}} & 
\multicolumn{1}{c|}{\textbf{Funcionamento do Software}} \\ \hline 
\endfirsthead

\hline
Caso de Uso Geral &  \\ \hline
Ator Principal & Portador da Deficiência Físico ou Motor \\ \hline
Ator Secundário & Responsável pelo usuário. \\ \hline
Resumo & Descreve o funcionamento do software. \\ \hline 
Pré-condições &  O computador com entrada de vídeo \\ \hline 
 Pós-condições&  \\ \hline
 Ações do Ator& Ações do Sistema \\ \hline
 1.	Executar o Software&  \\ \hline
 &  2.	Inicialização\\ \hline
 3.	Posicionamento do Usuário&  \\ \hline
 & 4.	Reconhecimento da face  \\ \hline
 & 5.	Feedback do reconhecimento \\ \hline
 & 6.	Rastreio da face \\ \hline
 & 7.	Controle do mouse \\ \hline
8.	Movimento da cabeça &  \\ \hline
 &  9.	Movimento do mouse\\ \hline
\end{longtable}


\begin{comment}
Exemplos do Template

\chapter{Exemplo do pacote Algorithm}
\label{Apx:B}

\begin{algorithm}[!h]
\caption{Estimador ML otimizado.}\label{Alg:MAXVER}
\begin{algorithmic}[1]
\STATE Inicializar o contador: $j\leftarrow 1$;%
\STATE Fixar o limiar de variação das estimativas: $e_{\mathrm{out}}\leftarrow 10^{-4}$;%
\STATE Fixar o número máximo de iterações: $N\leftarrow 1000$;%
\STATE Computar o ponto inicial: $\hat \gamma(0)$;%
\STATE Determinar o limiar inicial: $e_1 \leftarrow1000$;%
\STATE Estabelecer o valor inicial de $\alpha$: $\hat \alpha(0) \leftarrow -10^{-6}$;%
\WHILE{ $e_j \geq e_{\mathrm{out}}$ e $ j\leq M$}
    \STATE Solucionar $\hat \alpha_j\leftarrow {\arg \max}_{\alpha}\;{l_1(\alpha; \gamma_{j-1},\mathbf{z},n)}$;%
    \STATE Solucionar $\hat \gamma_j\leftarrow {\arg \max}_{\gamma}\;{l_2(\gamma; \alpha_j,\mathbf{z},n)}$;%
    \STATE $j\leftarrow j+1$
    \STATE Computar o critério de convergência: $e_j$;%
\ENDWHILE
\end{algorithmic}
\end{algorithm}
\end{comment}