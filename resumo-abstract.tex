\chapter*{Resumo}

\begin{singlespace}
{\fontsize{12pt}{\baselineskip} \selectfont \noindent
No Brasil cerca de 6,2\% da população apresenta algum tipo de deficiência, entre elas a física representa 1,3\% da população, a qual 46,8\% tem um grau intenso, segundo a {Pesquisa  Nacional de Saúde do IBGE de 2013}. Este trabalho apresenta o desenvolvimento de uma Tecnologia Assistiva, um software nomeado VisiUMouse. O VisiUMouse foi desenvolvido para ser uma solução que permite a acessibilidade ao uso do computador, por pessoas com deficiência física, principalmente nos membros superiores. Neste trabalho utilizou-se os conceitos de Visão Computacional e \textit{Machine Learning}, para a reconhecimento dos movimentos dos olhos e rastreamento da face da pessoa, com isso é possível o controle do movimento do cursor e o clique do \textit{mouse}, permitindo assim, que a interação com o computador seja realizada com os movimentos da cabeça (flexão e extensão). O VisiUMouse foi projetado com o requisito de não necessitar de \textit{internet} ou um \textit{hardware} externo, porém um requisito fundamental, é a existência de um dispositivo de captação de vídeo, como uma \textit{webcam}, que é geralmente nativa aos computadores. Desta forma desobriga a pessoa com deficiência motora a portar ou adquirir um outro dispositivo, facilitando assim, uma possibilidade de inclusão digital. Esse trabalho apresenta um experimento comparativo entre o VisiUMouse e \textit{mouse} tradicional com o objetivo de validar seu funcionamento como uma tecnologia que permite o controle do computador, assim como o \textit{mouse}, entre os resultados, o VisiUMouse teve uma assertividade de 99,63\%.

}
\end{singlespace}

\begin{singlespace}
\noindent \onehalfspacing
\textbf{Palavras chaves}: Visão Computacional. Tecnologia Assistiva. VisiUMouse. Machine Learning. Acessibilidade. OpenCV. Rastreamento de Rosto. Rastreamento de Olhos.
\end{singlespace}


\chapter*{Abstract}
\begin{singlespace}
{\fontsize{12pt}{\baselineskip} \selectfont \noindent
In Brazil, about 6,2\% of the population has some type of disability, among
they represent 1,3\% of the population, of which 46,8\% have an intense degree, according to the {Pesquisa  Nacional de Saúde do IBGE de 2013}. This paper presents the development of Assistive Technology, a software named VisiUMouse. VisiUMouse has been developed to be a solution that allows accessibility to the use of the computer by people with physical disabilities, especially in the higher. In this paper we used the concepts of Computer Vision and Machine Learning, for recognition of eye movements and face tracking control of the cursor movement and click mouse, thus allowing interaction with the computer to be performed with the movements of the head (flexion and extension). VisiUMouse is designed with the requirement of not needing internet or external hardware, but a requirement fundamental, is the existence of a video capture device, such as a webcam, which is usually native to computers. This person with a motor impairment to carry or acquire another device, facilitating thus, a possibility of digital inclusion. This paper presents an experiment comparison between VisiUMouse and traditional mouse in order to validate its functioning as a technology that allows control of the computer, as well as as the mouse, among the results, VisiUMouse had an assertiveness of 99,63\%.

}
\end{singlespace}

\begin{singlespace}
\noindent \onehalfspacing
\textbf{Keywords}: Computer Vision. Assistive Technology. VisiUMouse. Machine Learning. Accessibility. OpenCV. Face Tracking. Eye Tracking.
\end{singlespace}