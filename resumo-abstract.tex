\chapter*{Resumo}

\noindent
\onehalfspacing
Este relatório apresenta a tecnologia VisiUMouse, que é uma solução que permite a acessibilidade ao uso do computador por pessoas com deficiência motora, principalmente nos membros superiores. Esta Tecnologia Assistiva usa os conceitos de Visão Computacional e Machine Learning para reconhecimento dos olhos e rastreamento de rosto do usuário. Através de uma entrada de vídeo, que rastreia o movimento dos olhos, é possível o controle do movimento do cursor e cliques do mouse, permitindo assim o uso do computador apenas com o movimento da cabeça: proporcionando uma amplificação nas habilidades funcionais desses usuários e, conseqüentemente, promovendo a inclusão. O VisiUMouse é uma tecnologia que não necessidade de internet ou um hardware externo, porém é fundamental uma entrada de vídeo, como uma webcam, que é comumente nativa aos computadores. No Brasil existem cerca de 45,6 milhões de pessoas com deficiência, representando 23,9\% da população brasileira; deste total 7\% apresentam deficiência motora, segundo o censo do IBGE de 2010.


% * <tsixav@gmail.com> 2018-03-01T12:00:53.184Z:
% 
% 01/03/2018
% versão antiga do resumo:
% Este relatório apresenta a tecnologia VisiUMouse, que é uma solução que permite a acessibilidade ao uso do computador por pessoas com deficiência motora nos membros superiores. Esta Tecnologia Assistiva usa os conceitos de Visão Computacional para reconhecimento e rastreamento de rostos, usando o olho como ponto de referência. Através da entrada de vídeo, que rastreia o movimento dos olhos, permitindo o controle do movimento do cursor do mouse, proporcionando uma amplificação nas habilidades funcionais desses usuários e, conseqüentemente, promovendo a inclusão.
% 
% ^.

\noindent
\onehalfspacing
\textbf{Palavras chaves}: Visão de Computador, Tecnologia Assistiva, VisiUMouse, Inteligência Artificial, Acessibilidade, Computação Ubíqua, OpenCV, Rastreamento de Rosto, Rastreamento de Olhos.





\chapter*{Abstract}
\onehalfspacing
\noindent This report introduces the VisiUMouse technology, which is a solution that allows accessibility to computer use by people with motor disabilities, especially in the upper limbs. This Assistive Technology uses the concepts of Computer Vision and Machine Learning for eye recognition and user face tracking. Through a video input, which tracks the movement of the eyes, it is possible to control the movement of the cursor and mouse clicks, thus allowing the use of the computer only with the movement of the head: providing an amplification in the functional abilities of these users and, consequently, promoting inclusion. VisiUMouse is a technology that does not need the internet or an external hardware, however it is fundamental a video input, such as a webcam, which is commonly native to computers. In Brazil there are about 45.6 million people with disabilities, representing 23.9 \% of the Brazilian population; of this total 7 \% have motor impairment, according to the 2010 IBGE census.

% * <tsixav@gmail.com> 2018-03-01T12:36:24.434Z:
% 
% 01/03/2018
% versão antiga:
% 
% This report introduces the VisiUMouse technology, which is a a solution that enables accessibility to computer's use by people with upper limbs motor disability. This Assistive Technology uses the concepts of Computational Vision to recognition and face tracking, using the eye as reference point. Through video input, that tracks the eye movement, we control the mouse cursor by webcam, providing an amplification in the functional abilities of these users and consequently, promoting inclusion. The VisiUMouse was evaluated following the Fitts's law protocol, which involves common pointing, select and click tasks, used the metrics to verify interaction with the computer.
% 
% ^.


\noindent
\onehalfspacing
\textbf{Keywords}: Computer Vision, Assistive Technology, VisiUMouse, Artificial Intelligence, Accessibility, Ubiquitous Computing, OpenCV, Face Tracking, Eye Tracking