\chapter*{Resumo}

{\fontsize{12pt}{\baselineskip} \selectfont \noindent \setstretch{1.5}
No Brasil cerca de 6,2\% da população apresenta algum tipo de deficiência, entre elas a física representa 1,3\% da população, a qual 46,8\% tem um grau intenso, segundo a {Pesquisa  Nacional de Saúde do IBGE de 2013}. Este trabalho apresenta o desenvolvimento de uma Tecnologia Assistiva, um software nomeado VisiUMouse. O VisiUMouse foi desenvolvido para ser uma solução que permite a acessibilidade ao uso do computador, por pessoas com deficiência física, principalmente nos membros superiores. Neste trabalho utilizou-se os conceitos de Visão Computacional e \textit{Machine Learning}, para reconhecimento dos movimentos dos olhos e rastreamento da face da pessoa. Com isso é possível o controle do movimento do cursor e o clique do \textit{mouse}, permitindo assim, que a interação com o computador seja realizada com os movimentos da cabeça (flexão e extensão). O VisiUMouse foi projetado com o objetivo de não necessitar de \textit{internet} ou de \textit{hardware} externo. Porém, um requisito fundamental é a existência de um dispositivo que permita a captação de vídeo, como uma \textit{webcam}, que é geralmente nativa aos computadores. Desta forma a solução desobriga a pessoa com deficiência motora a portar ou adquirir um outro dispositivo, facilitando assim, a possibilidade de inclusão digital. Esse trabalho apresenta ainda um experimento comparativo entre o VisiUMouse e o \textit{mouse} tradicional com o objetivo de validar seu funcionamento como uma tecnologia que permite o controle do computador. Entre os resultados alcançados, destaca-se o fato do VisiUMouse possuir uma assertividade de 99,63\%.
}
\\

\noindent \onehalfspacing \setstretch{1.5}
\textbf{Palavras chaves}: Visão Computacional. Tecnologia Assistiva. VisiUMouse. Machine Learning. Acessibilidade. OpenCV. Rastreamento de Rosto. Rastreamento de Olhos.


\chapter*{Abstract}
{\fontsize{12pt}{\baselineskip} \selectfont \noindent \setstretch{1.5}
In Brazil, about 6,2\% of the population has some type of disability, among them physics represents 1,3\% of the population, which 46,8\% has an intense degree, according to the {Pesquisa  Nacional de Saúde do IBGE de 2013}. This paper presents the development of an Assistive Technology, a software named VisiUMouse. VisiUMouse was developed to be a solution that allows accessibility to the use of the computer by people with physical disabilities, mainly in the upper limbs. In this paper, the concepts of Computer Vision and Machine Learning were used to recognize the movements of the eyes and to trace the person's face. With this it is possible to control the movement of the cursor and the click of the mouse, thus allowing interaction with the computer to be performed with the movements of the head (flexion and extension). VisiUMouse is designed with the objective of not needing internet or external hardware. However, a fundamental requirement is the existence of a device that allows the capture of video, such as a webcam, which is usually native to computers. In this way the solution releases the person with motor deficiency to carry or acquire another device, thus facilitating the possibility of digital inclusion. This paper presents a comparative experiment between VisiUMouse and the traditional mouse with the objective of validating its operation as a technology that allows the control of the computer. Among the results achieved, we highlight the fact that VisiUMouse has an assertiveness of 99,63\%.
}
\\

\noindent \onehalfspacing \setstretch{1.5}
\textbf{Keywords}: Computer Vision. Assistive Technology. VisiUMouse. Machine Learning. Accessibility. OpenCV. Face Tracking. Eye Tracking.