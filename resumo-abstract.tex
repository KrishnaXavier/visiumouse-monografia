\chapter*{Resumo}

\begin{singlespace}
{\fontsize{12pt}{\baselineskip} \selectfont \noindent
No Brasil cerca de 6,2 \% da população tem deficiência, dentre os tipos de deficiência a deficiência física representa 1,3\% da população, a qual 46,8\% tem um grau intenso, segundo a Pesquisa  Nacional de Saúde do IBGE de 2013. Este trabalho apresenta o desenvolvimento de uma Tecnologia Assistiva, um software nomeado VisiuMouse. O VisiuMouse foi desenvolvido para ser uma solução que permita a acessibilidade ao uso do computador, por pessoas com deficiência física, principalmente nos membros superiores. Sendo desta forma, considerado como uma Tecnologia Assistiva. Neste trabalho utilizou-se os conceitos de Visão Computacional e \textit{Machine Learning}, para a funcionalidade de reconhecimento dos movimentos dos olhos e rastreamento da face da pessoa. Através de uma entrada de vídeo, que rastreia o movimento dos olhos, é possível o controle do movimento do cursor e o clique do mouse, permitindo assim, que a interação com a interface gráfica seja realizada com os movimentos da cabeça (flexão e extensão). O VisiUMouse foi projetado com o requisito de não necessitar, necessariamente, de internet ou um hardware externo, já que não é um sistema web, porém um requisito fundamental, é a existência de um dispositivo de captação de vídeo, como uma \textit{webcam}, que é geralmente nativa aos computadores. Desta forma desobriga a pessoa com deficiência motora a portar ou adquirir um outro dispositivo, facilitando assim, uma possibilidade de inclusão digital .
}
\end{singlespace}

\begin{singlespace}
\noindent \onehalfspacing
\textbf{Palavras chaves}: Visão de Computador. Tecnologia Assistiva. VisiUMouse. Inteligência Artificial. Acessibilidade. Computação Ubíqua. OpenCV. Rastreamento de Rosto. Rastreamento de Olhos.
\end{singlespace}


\chapter*{Abstract}
\begin{singlespace}
{\fontsize{12pt}{\baselineskip} \selectfont \noindent
In Brazil, about 6,2\% of the population has a disability, among the types of disability the physical disability represents 1,3\% of the population, which 46,8\% has an intense degree, according to the Pesquisa  Nacional de Saúde of IBGE of 2013. This work presents the technology VisiUMouse, which is a solution that allows the accessibility to the use of the computer by people with physical deficiency, mainly in the upper members. This Assistive Technology uses the concepts of Computer Vision and Machine Learning for eye recognition and user face tracking. Through a video input, which the movement of the eyes, it is possible to control the movement of the cursor and mouse clicks, thus allowing the use of the computer only with the movement of the head, providing an amplification in the functional abilities of these users and, consequently, promoting the inclusion. VisiUMouse is a technology that does not require the internet or external hardware, however a video input, such as a webcam, is usually a native computer.
}
\end{singlespace}

\begin{singlespace}
\noindent \onehalfspacing
\textbf{Keywords}: Computer Vision. Assistive Technology. VisiUMouse. Artificial Intelligence. Accessibility. Ubiquitous Computing. OpenCV. Face Tracking. Eye Tracking
\end{singlespace}