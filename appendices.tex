\chapter{DOCUMENTAÇÃO CASO DE USO}
\label{Apx:A}

%antigo
% \begin{longtable}{|l|l|}
% \caption{Documentação Caso de Uso.} \label{tab:long} \\
% \hline 
% \multicolumn{1}{|c|}{\textbf{Nome do Caso de Uso}} & 
% \multicolumn{1}{c|}{\textbf{Funcionamento do Software}} \\ \hline 
% \endfirsthead
% \hline
% Caso de Uso Geral &  \\ \hline
% Ator Principal & Portador da Deficiência Físico ou Motor \\ \hline
% Ator Secundário & Responsável pelo usuário. \\ \hline
% Resumo & Descreve o funcionamento do software. \\ \hline 
% Pré-condições &  O computador com entrada de vídeo \\ \hline 
%  Pós-condições&  \\ \hline
%  Ações do Ator& Ações do Sistema \\ \hline
%  1.	Executar o Software&  \\ \hline
%  &  2.	Inicialização\\ \hline
%  3.	Posicionamento do Usuário&  \\ \hline
%  & 4.	Reconhecimento da face  \\ \hline
%  & 5.	Feedback do reconhecimento \\ \hline
%  & 6.	Rastreio da face \\ \hline
%  & 7.	Controle do mouse \\ \hline
% 8.	Movimento da cabeça &  \\ \hline
%  &  9.	Movimento do mouse\\ \hline
% \end{longtable}

%------------------------- fazendo

\begin{longtable}{|l|l|}
\caption{Documentação Caso de Uso \textit{Iniciar aplicação}.} \label{tab:dcu-1} \\
\hline 
\multicolumn{1}{|c|}{\textbf{Nome do Caso de Uso}} & 
\multicolumn{1}{c|}{\textbf{Funcionamento do Software}} \\ \hline 
\endfirsthead
\hline
Caso de Uso Geral &  \\ \hline
Ator Principal & Usuário, responsável\\ \hline
Ator Secundário &  \\ \hline
Resumo & Usuário/responsável iniciando a aplicação. \\ \hline 
Pré-condições &  O Software deve ter sido executado. \\ \hline 
Pós-condições &  Mostrar a tela de captura de vídeo. \\ \hline
 Ações do Ator& Ações do Sistema \\ \hline
 1.	Clicar em iniciar na tela principal.&  \\ \hline
\end{longtable}

\begin{longtable}{|l|l|}
\caption{Documentação Caso de Uso \textit{Configurar}.} \label{tab:dcu-1} \\
\hline 
\multicolumn{1}{|c|}{\textbf{Nome do Caso de Uso}} & 
\multicolumn{1}{c|}{\textbf{Funcionamento do Software}} \\ \hline 
\endfirsthead
\hline
Caso de Uso Geral &  \\ \hline
Ator Principal & Usuário\\ \hline
Ator Secundário & \\ \hline
Resumo & Usuário fazendo configuração. \\ \hline 
Pré-condições &  Estar na parte de configuração. \\ \hline 
Pós-condições &  Configuração. \\ \hline
 Ações do Ator& Ações do Sistema \\ \hline
 1.	Clicar em uma das configurações.&  \\ \hline
 2.	Configurar um parâmetro.&  \\ \hline
 & 3. Envia a novo parâmetro para o sistema. \\ \hline
\end{longtable}

\begin{longtable}{|l|l|}
\caption{Documentação Caso de Uso \textit{Reconhecer olhos}.} \label{tab:dcu-1} \\
\hline 
\multicolumn{1}{|c|}{\textbf{Nome do Caso de Uso}} & 
\multicolumn{1}{c|}{\textbf{Funcionamento do Software}} \\ \hline 
\endfirsthead
\hline
Caso de Uso Geral &  \\ \hline
Ator Principal & Usuário\\ \hline
Ator Secundário & \\ \hline
Resumo & Sistema rastreando os olhos do usuário. \\ \hline 
Pré-condições &  Aplicação iniciada. \\ \hline 
Pós-condições &  Leitura da posição dos olhos. \\ \hline
 Ações do Ator& Ações do Sistema \\ \hline
 1.	Clicar em calibrar.&  \\ \hline
 & 2. Rastrear a posição dos olhos do usuário. \\ \hline
\end{longtable}

\begin{longtable}{|l|l|}
\caption{Documentação Caso de Uso \textit{Definir olhos principais}.} \label{tab:dcu-1} \\
\hline 
\multicolumn{1}{|c|}{\textbf{Nome do Caso de Uso}} & 
\multicolumn{1}{c|}{\textbf{Funcionamento do Software}} \\ \hline 
\endfirsthead
\hline
Caso de Uso Geral &  \\ \hline
Ator Principal & Usuário\\ \hline
Ator Secundário & \\ \hline
Resumo & Define os olhos do usuário. \\ \hline 
Pré-condições &  Aplicação calibrada. \\ \hline 
Pós-condições &  \\ \hline
 Ações do Ator& Ações do Sistema \\ \hline
 & 1. Identifica os olhos. \\ \hline
\end{longtable}

\begin{longtable}{|l|l|}
\caption{Documentação Caso de Uso \textit{Controlar mouse}.} \label{tab:dcu-1} \\
\hline 
\multicolumn{1}{|c|}{\textbf{Nome do Caso de Uso}} & 
\multicolumn{1}{c|}{\textbf{Funcionamento do Software}} \\ \hline 
\endfirsthead
\hline
Caso de Uso Geral &  \\ \hline
Ator Principal & Usuário\\ \hline
Ator Secundário & \\ \hline
Resumo & Controle do mouse. \\ \hline 
Pré-condições &  Olhos identificados. \\ \hline 
Pós-condições &  \\ \hline
Ações do Ator& Ações do Sistema \\ \hline
 1. Controle do mouse. &  \\ \hline
\end{longtable}


\begin{comment}
Exemplos do Template

\chapter{Exemplo do pacote Algorithm}
\label{Apx:B}

\begin{algorithm}[!h]
\caption{Estimador ML otimizado.}\label{Alg:MAXVER}
\begin{algorithmic}[1]
\STATE Inicializar o contador: $j\leftarrow 1$;%
\STATE Fixar o limiar de variação das estimativas: $e_{\mathrm{out}}\leftarrow 10^{-4}$;%
\STATE Fixar o número máximo de iterações: $N\leftarrow 1000$;%
\STATE Computar o ponto inicial: $\hat \gamma(0)$;%
\STATE Determinar o limiar inicial: $e_1 \leftarrow1000$;%
\STATE Estabelecer o valor inicial de $\alpha$: $\hat \alpha(0) \leftarrow -10^{-6}$;%
\WHILE{ $e_j \geq e_{\mathrm{out}}$ e $ j\leq M$}
    \STATE Solucionar $\hat \alpha_j\leftarrow {\arg \max}_{\alpha}\;{l_1(\alpha; \gamma_{j-1},\mathbf{z},n)}$;%
    \STATE Solucionar $\hat \gamma_j\leftarrow {\arg \max}_{\gamma}\;{l_2(\gamma; \alpha_j,\mathbf{z},n)}$;%
    \STATE $j\leftarrow j+1$
    \STATE Computar o critério de convergência: $e_j$;%
\ENDWHILE
\end{algorithmic}
\end{algorithm}
\end{comment}