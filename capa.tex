\thispagestyle{empty}

\vfill
 \begin{center}

%  	 Atividade 3
%     \begin{figure}[t]
%      \centering
%             \includegraphics[width=8cm]{figures/logo-ifsul-pelotas.jpg}\\[-0.1in]
%      \end{figure}

		\begin{figure}[t]
			\centering
			\includegraphics{figures/logo-ifsul-pelotas-tsi.jpg}	
		\end{figure}

 {\large\bfseries INSTITUTO FEDERAL DE EDUCAÇÃO, CIÊNCIA E TECNOLOGIA SUL-RIO-GRANDENSE} \\

    \vspace*{1in}
    \begin{large} \bfseries KRISHNA FERREIRA XAVIER\end{large}\\[0.4in]

    \vspace*{4cm}
    \noindent \\
    \large\bfseries{VisiUMouse: Uma Tecnologia de Visão Computacional Ubíqua para pessoas com Deficiência Motora} \\
    \vfill
    \large\bfseries{ PELOTAS \\ 2018}
\end{center}

\normalsize

\begin{titlepage}
\vfill
\begin{center}

    {\large KRISHNA FERREIRA XAVIER\\}
    \vspace{2cm}
    {\Large \textsc{VisiUMouse: Uma Tecnologia de Visão Computacional Ubíqua para pessoas com Deficiência Motora}\\}
    \vspace{1cm}
    \hspace{.45\linewidth}
    \begin{minipage}{.50\linewidth}

           Trabalho de Conclusão do VI semestre do Curso Superior de Tecnologia em Sistemas para Internet, do Instituto Federal Sul- rio-grandense.

            \vspace{0.5 cm}

            Área de pesquisa: Human-centered computing Accessibility technologies, Human-centered computing Interaction design, Human-centered computing Ubiquitous and mobile computing, Human-centered computing Accessibility, Human-centered computing Human computer interaction (HCI)

            \vspace{0.5 cm}

            Orientador: Rafael Cunha Cardoso
    
    \end{minipage}

    \vspace{2cm}
    \vfill
    {\large Pelotas\\ 2018}
\end{center}

\end{titlepage}

\begin{comment}
\begin{folhadeaprovacao}
\setlength{\ABNTsignthickness}{0.2pt}
\setlength{\ABNTsignskip}{1.7cm}

\begin{center}
\includegraphics[width=2.5cm]{figures/brasao_republica.eps}\\
%\includegraphics[width=5cm]{figures/IF_logo.eps}\\ %outros brasões
%\includegraphics[width=5cm]{figures/IF_logo2.eps}\\%outros brasões

            {INSTITUTO FEDERAL DE EDUCAÇÃO CIÊNCIA E TECNOLOGIA SUL-RIO-GRANDENSE} \\
            { \{NOME DA COORDENAÇÃO\} }  \\

    \vspace{1.5cm}
                                    {KRISHNA FERREIRA XAVIER}\\
    \bfseries{}
\end{center}

Este(a) preencher se Trabalho de Conclusão de Curso (TCC ou Monografia ou
Dissertação foi apresentado(a) em preencher o dia de preencher o mês de
preencher o ano como requisito parcial para a obtenção do título de preencher
se Bacharel ou Tecnólogo ou Especialista ou Mestre em preencher o nome
do curso. O(a) candidato(a) foi arguido pela Banca Examinadora composta
pelos professores abaixo assinados. Após deliberação, a Banca Examinadora
considerou o trabalho aprovado.

    \vspace{0.15cm}
    \assinatura{Orientador: Rafael Cunha Cardoso \\ Instituto Federal de Educação Ciência e Tecnologia Sul-Rio-Grandense - IFSul - Pelotas}
    \assinatura{Prof. \{NOME\} \\ Instituto Federal de Educação Ciência e Tecnologia Sul-Rio-Grandense - IFSul - Pelotas}
    \assinatura{Prof. \{NOME\} \\ Instituto Federal de Educação Ciência e Tecnologia Sul-Rio-Grandense - IFSul - Pelotas}
    \assinatura{Prof. \{NOME\} \\ Instituto Federal de Educação Ciência e Tecnologia Sul-Rio-Grandense - IFSul - Pelotas}
    \vspace{0.15cm}%\vfill

    \begin{center}
        Pelotas, \{DATA\}
    \end{center}
\end{folhadeaprovacao}
\end{comment}

\chapter*{Agradecimentos}
\vspace*{5cm}
\hfill Este trabalho foi fruto do conhecimento e das oportunidades proporcionadas pelos orientadores, grupos de pesquisas, projetos e professores presentes em toda minha vida acadêmica, e da estrutura que IFSul Campus Pelotas provém. E por fim agradecimentos para os familiares, amigos e colegas.\\

%\thispagestyle{empty}

%\include{epigrafe}
\begin{flushright}
\begin{minipage}[r]{10cm}
\vspace{18cm}
``tudo é tão existencial quanto possivel''.
\begin{flushright}
Krishna Xavier
\end{flushright}
\end{minipage}
\end{flushright}