\chapter*{Resumo}

\begin{singlespace}
{\fontsize{12pt}{\baselineskip} \selectfont \noindent
No Brasil cerca de 6,2 \% da população tem deficiência, dentre os tipos de deficiência a deficiência física representa 1,3\% da população, a qual 46,8\% tem um grau intenso, segundo a Pesquisa  Nacional de Saúde do IBGE de 2013. Este trabalho apresenta o desenvolvimento de uma Tecnologia Assistiva, um software nomeado VisiUMouse. O VisiUMouse foi desenvolvido para ser uma solução que permita a acessibilidade ao uso do computador, por pessoas com deficiência física, principalmente nos membros superiores. Neste trabalho utilizou-se os conceitos de Visão Computacional e \textit{Machine Learning}, para a reconhecimento dos movimentos dos olhos e rastreamento da face da pessoa, com isso é possível o controle do movimento do cursor e o clique do \textit{mouse}, permitindo assim, que a interação com a interface gráfica seja realizada com os movimentos da cabeça (flexão e extensão). O VisiUMouse foi projetado com o requisito de não necessitar, necessariamente, de internet ou um \textit{hardware} externo, porém um requisito fundamental, é a existência de um dispositivo de captação de vídeo, como uma \textit{webcam}, que é geralmente nativa aos computadores. Desta forma desobriga a pessoa com deficiência motora a portar ou adquirir um outro dispositivo, facilitando assim, uma possibilidade de inclusão digital. Esse trabalho apresenta um experimento comparativo entre o VisiUMouse e \textit{mouse} tradicional com o objetivo de validar seu funcionamento como uma tecnologia que permite o controle do computador, assim como o \textit{mouse}, entre os resultados a assertividade teve um diferença menor do que 1\% (0,37\%), do \textit{mouse} em relação ao VisiUMouse.
}
\end{singlespace}

\begin{singlespace}
\noindent \onehalfspacing
\textbf{Palavras chaves}: Visão Computacional. Tecnologia Assistiva. VisiUMouse. Machine Learning. Acessibilidade. OpenCV. Rastreamento de Rosto. Rastreamento de Olhos.
\end{singlespace}


\chapter*{Abstract}
\begin{singlespace}
{\fontsize{12pt}{\baselineskip} \selectfont \noindent
In Brazil, about 6,2\% of the population has a disability, among the types of disability, physical disability represents 1,3\% of the population, of which 46,8\% have an intense degree, according to the Pesquisa  Nacional de Saúde of IBGE of 2013. This paper presents the development of Assistive Technology, a software named VisiUMouse. VisiUMouse was developed to be a solution that allows the accessibility to the use of the computer, by people with physical deficiency, mainly in the upper members. In this paper the concepts of Computational Vision and Machine Learning were used to recognize the movements of the eyes and to trace the person's face, so that it is possible to control the movement of the cursor and the click of the mouse, thus allowing the interaction with the graphic interface being performed with the movements of the head (flexion and extension). VisiUMouse is designed with the requirement of not necessarily needing internet or external hardware, but a fundamental requirement is the existence of a video capture device, such as a webcam, which is usually native to computers. In this way it releases the person with motor disability to carry or acquire another device, thus facilitating a possibility of digital inclusion. This paper presents a comparative experiment between the VisiUMouse and traditional mouse with the objective of validating its operation as a technology that allows the control of the computer, as well as the mouse, among the results the assertiveness had a difference smaller than 1\% (0,37\%), mouse over VisiUMouse.

}
\end{singlespace}

\begin{singlespace}
\noindent \onehalfspacing
\textbf{Keywords}: Computer Vision. Assistive Technology. VisiUMouse. Machine Learning. Accessibility. OpenCV. Face Tracking. Eye Tracking.
\end{singlespace}