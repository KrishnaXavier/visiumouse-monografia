\chapter*{Resumo}

\begin{singlespace}
{\fontsize{12pt}{\baselineskip} \selectfont \noindent
No Brasil cerca de 6,2 \% da população tem deficiência, dentre os tipos de deficiência a deficiência física representa 1,3\% da população, a qual 46,8\% tem um grau intenso, que proíbe realizar atividades habituais, segundo a Pesquisa  Nacional de Saúde do IBGE de 2013. Este trabalho apresenta a tecnologia VisiUMouse, que é uma solução que permite a acessibilidade ao uso do computador por pessoas com deficiência física, principalmente nos membros superiores. Esta Tecnologia Assistiva usa os conceitos de Visão Computacional e \textit{Machine Learning} para reconhecimento dos olhos e rastreamento de rosto do usuário. Através de uma entrada de vídeo, que rastreia o movimento dos olhos, é possível o controle do movimento do cursor e cliques do mouse, permitindo assim o uso do computador apenas com o movimento da cabeça, proporcionando uma amplificação nas habilidades funcionais desses usuários e, conseqüentemente, promovendo a inclusão. O VisiUMouse é uma tecnologia que não necessita de internet ou um hardware externo, porém é fundamental uma entrada de vídeo, como uma \textit{webcam}, que é comumente nativa aos computadores. 
}
\end{singlespace}

\begin{singlespace}
\noindent \onehalfspacing
\textbf{Palavras chaves}: Visão de Computador. Tecnologia Assistiva. VisiUMouse. Inteligência Artificial. Acessibilidade. Computação Ubíqua. OpenCV. Rastreamento de Rosto. Rastreamento de Olhos.
\end{singlespace}


\chapter*{Abstract}
\begin{singlespace}
{\fontsize{12pt}{\baselineskip} \selectfont \noindent
In Brazil, about 6,2\% of the population has a disability, among the types of disability the physical disability represents 1,3\% of the population, which 46,8\% has an intense degree, which prohibits habitual activities, according to the Pesquisa  Nacional de Saúde of IBGE of 2013. This work presents the technology VisiUMouse, which is a solution that allows the accessibility to the use of the computer by people with physical deficiency, mainly in the upper members. This Assistive Technology uses the concepts of Computer Vision and Machine Learning for eye recognition and user face tracking. Through a video input, which the movement of the eyes, it is possible to control the movement of the cursor and mouse clicks, thus allowing the use of the computer only with the movement of the head, providing an amplification in the functional abilities of these users and, consequently, promoting the inclusion. VisiUMouse is a technology that does not require the internet or external hardware, however a video input, such as a webcam, is usually a native computer.
}
\end{singlespace}

\begin{singlespace}
\noindent \onehalfspacing
\textbf{Keywords}: Computer Vision. Assistive Technology. VisiUMouse. Artificial Intelligence. Accessibility. Ubiquitous Computing. OpenCV. Face Tracking. Eye Tracking
\end{singlespace}